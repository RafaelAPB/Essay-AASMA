%\section{Introduction}
\section{Agents}
An agent is a system that tries to fulfill a set of goals in a complex environment. Agents needs to be autonomous and adaptive, in order to cope with the vicissitudes of environments. In a multi-agent scenario, agents can compete, being their interactions described by a game encounter. For maximising the utility, a mathematical function which ranks different options according to their utility to an individual, an agent can use the environment.
\section{Game Encounters}
A game is a competitive activity, where players compete with each other (interactions), according to defined rules (strategy) \cite{aulas}
. Players can only take certain actions defined by the environment.  Assuming that agents simultaneously choose an action to perform, the result of their actions depends on the combination of actions. The environment is then changed, according to the cumulative set of actions performed by all agents. This fact arises one question: if all agents influence the environment, and if all agents want to maximise their utility, how should they act? The choose of an action has some relevant issues, including but not limited to the nature of the goal (what kind, static or dynamic), scaling up, reusability, the depth of understanding of actions leading to emergent behaviours and the relation between perceptions and actions \cite{Maes:1993:MAA:1668014.1668022}. Game theory studies interactions between rational agents who aim to maximise their utility. Agents can negotiate. An outcome is Pareto efficient if no other
outcome improves a player’s utility without making someone else worse off \cite{aulas}.

\subsection{Deception Mechanisms}
Deception is often used by humans to raise the probability of success when negotiating. The ability of an agent to negotiate efficiently in the presence of conflicting goals is related with the information that the adversary holds. 

Let us assume an encounter between two competing agents, with the possibility of them to use the following deception mechanisms \textit{i.} hidden actions, \textit{ii.} hidden utilities and \textit{iii.} decoy actions. Deception techniques can occur on inter-agent negotiation with incomplete information, as negotiation typically assume trustworthy agents, which is not always the case \cite{inproceedingsa}. Let us assume a bi-matrix \textit{b}, where \textit{agent i} and \textit{agent j} have different goals, g$_i$ and g$_j$, respectively. Both agents want to transform the world from the initial to a state \textit{s$_i$} or \textit{s$_j$} that satisfies its goal. As agents co-exist and might interfeer 

\section{Conclusions}
This paragraph will end the body of this sample document.
Remember that you might still have Acknowledgments or
Appendices; brief samples of these
follow.  There is still the Bibliography to deal with; and
we will make a disclaimer about that here: with the exception
of the reference to the \LaTeX\ book, the citations in
this paper are to articles which have nothing to
do with the present subject and are used as
examples only.
%\end{document}  % This is where a 'short' article might terminate



\appendix
%Appendix A
\section{Headings in Appendices}
The rules about hierarchical headings discussed above for
the body of the article are different in the appendices.
In the \textbf{appendix} environment, the command
\textbf{section} is used to
indicate the start of each Appendix, with alphabetic order
designation (i.e., the first is A, the second B, etc.) and
a title (if you include one).  So, if you need
hierarchical structure
\textit{within} an Appendix, start with \textbf{subsection} as the
highest level. Here is an outline of the body of this
document in Appendix-appropriate form:
\subsection{Introduction}
\subsection{The Body of the Paper}
\subsubsection{Type Changes and  Special Characters}
\subsubsection{Math Equations}
\paragraph{Inline (In-text) Equations}
\paragraph{Display Equations}
\subsubsection{Citations}
\subsubsection{Tables}
\subsubsection{Figures}
\subsubsection{Theorem-like Constructs}
\subsubsection*{A Caveat for the \TeX\ Expert}
\subsection{Conclusions}
\subsection{References}
Generated by bibtex from your \texttt{.bib} file.  Run latex,
then bibtex, then latex twice (to resolve references)
to create the \texttt{.bbl} file.  Insert that \texttt{.bbl}
file into the \texttt{.tex} source file and comment out
the command \texttt{{\char'134}thebibliography}.
% This next section command marks the start of
% Appendix B, and does not continue the present hierarchy
\section{More Help for the Hardy}

Of course, reading the source code is always useful.  The file
\path{acmart.pdf} contains both the user guide and the commented
code.

\begin{acks}
  The authors would like to thank Dr. Yuhua Li for providing the
  matlab code of  the \textit{BEPS} method. 

  The authors would also like to thank the anonymous referees for
  their valuable comments and helpful suggestions. The work is
  supported by the \grantsponsor{GS501100001809}{National Natural
    Science Foundation of
    China}{http://dx.doi.org/10.13039/501100001809} under Grant
  No.:~\grantnum{GS501100001809}{61273304}
  and~\grantnum[http://www.nnsf.cn/youngscientsts]{GS501100001809}{Young
    Scientsts' Support Program}.

\end{acks}
